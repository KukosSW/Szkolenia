\section{Dodatek}

\subsection{Benchmark} \label{dodatek:benchmark}
\begin{lstlisting}[language=C,style=C99]
#include <time.h>
#include <stdio.h>
#include <stdlib.h>

#define CLOCK_TYPE CLOCK_MONOTONIC

#define MEASURE_FUNCTION(func, label) \
    do { \
        struct timespec start; \
        struct timespec end; \
        double timeTaken; \
        \
        clock_gettime(CLOCK_TYPE, &start); \
        (void)func; \
        clock_gettime(CLOCK_TYPE, &end); \
        \
        timeTaken = (double)(end.tv_sec - start.tv_sec) * 1e9; \
        timeTaken = (double)(timeTaken + (double)(end.tv_nsec - start.tv_nsec)) * 1e-9; \
        \
        printf(label " time = %lf[s]\n", timeTaken); \
    } while (0)
    
extern int fib(int n);

int main(void)
{
    MEASURE_FUNCTION(fib(45), "Fibonacci(45)");
    
    return 0;
}
\end{lstlisting}

\subsection{Makefile} \label{dodatek:makefile}
Zakładamy że posiadamy następującą strukturę katalogów:
\begin{itemize}
    \item \textbf{./} - katalog głowny projektu gdzie trzymamy plik Makefile, README oraz inne pliki konfiguracyjne.
    \item \textbf{./src} - katalog zawierający kod źródłowy wszystkich plików prócz pliku zawierającego funkcję main.
    \item \textbf{./app} - katalog zawierający kod źródłowy pliku z funkcją main.
    \item \textbf{./inc} - katalog zawierający pliki nagłówkowe.
\end{itemize}

\clearpage
Wtedy przykładowy plik Makefile może wyglądać następująco.

\begin{lstlisting}[language=make, style=mymake]
# Shell commands
RM          := rm -rf

# Compiler setting (default it will be gcc)
CC          ?= gcc

# Enable max optimization
CC_OPT      := -O3

# Maybe some flags are duplicated, but who cares
CC_WARNINGS := -Wall -Wextra -pedantic -Wcast-align \
               -Winit-self -Wmissing-include-dirs \
               -Wredundant-decls -Wshadow -Wstrict-overflow=5 \
               -Wundef -Wwrite-strings -Wpointer-arith \
               -Wmissing-declarations -Wuninitialized \
               -Wold-style-definition -Wstrict-prototypes \
               -Wmissing-prototypes -Wswitch-default \
               -Wbad-function-cast -Wnested-externs \
               -Wconversion -Wunreachable-code \

ifeq ($(CC),$(filter $(CC), gcc cc))
CC_SYM      :=  -rdynamic
CC_STD      :=  -std=gnu99
else ifeq ($(CC),clang)
CC_SYM      :=  -Wl,--export-dynamic
CC_WARNINGS +=  -Wgnu -Weverything -Wno-newline-eof \
                -Wno-unused-command-line-argument \
                -Wno-reserved-id-macro -Wno-documentation \
                -Wno-documentation-unknown-command \
                -Wno-padded
CC_STD      :=  -std=c99
endif

CC_FLAGS    := $(CC_STD) $(CC_WARNINGS) $(CC_OPT) $(CC_SYM)

PROJECT_DIR := $(shell pwd)

# To enable verbose mode type make V=1
ifeq ("$(origin V)", "command line")
  VERBOSE = $(V)
endif

ifndef VERBOSE
  VERBOSE = 0
endif

ifeq ($(VERBOSE),1)
  Q =
else
  Q = @
endif

define print_info
	$(if $(Q), @echo "$(1)")
endef

define print_make
	$(if $(Q), @echo "[MAKE]     $(1)")
endef

define print_cc
	$(if $(Q), @echo "[CC]      $(1)")
endef

define print_bin
	$(if $(Q), @echo "[BIN]     $(1)")
endef


IDIR := $(PROJECT_DIR)/inc
SDIR := $(PROJECT_DIR)/src
ADIR := $(PROJECT_DIR)/app

SRCS := $(wildcard $(SDIR)/*.c) $(wildcard $(ADIR)/*.c)
OBJS := $(SRCS:%.c=%.o)
DEPS := $(wildcard $(IDIR)/*.h)

# Put here all needed libraries like math, pthread etc
LIBS := -lm

# Type here name of your output file
EXEC := $(PROJECT_DIR)/main.out

all: $(EXEC)

%.o: %.c
	$(call print_cc, $<)
	$(Q)$(CC) $(CC_FLAGS) -I$(IDIR) -c $< -o $@

$(EXEC): $(OBJS)
	$(call print_bin, $@)
	$(Q)$(CC) $(CC_FLAGS) -I$(IDIR) $(OBJS) $(LIBS) -o $@

clean:
	$(call print_info,Cleaning)
	$(Q)$(RM) $(OBJS)
	$(Q)$(RM) $(EXEC)

\end{lstlisting}
\clearpage

\subsection{Przykład weryfikatora jednowątkowego w języku promela} \label{dodatek:promelaST}
Model implementuje proces, w którym losujemy zawsze 2 kule. Jeśli kule są tego samego koloru to dokładamy kulę czarną do puli, w przeciwnym razie dokładamy kule białą. \\
Model ma za zadanie sprawdzić czy gdy zaczynamy z 150 kulami czarnymi i 75 kulami białymi to zawsze skończymy z jedną kulą białą? \\
Prawdopodobieństwo wylosowania się kul jest jednostajne.
\begin{lstlisting}[language=Promela,style=myPromela]
/*
    System:
    We have 150 black balls and 75 white balls

    Process:
    Get 2 balls
    if (balls have the same color)
        add black ball to urn
    else
        add white ball to urn

    CHECK:
        Always at the end we have 1 white ball

    Program:
        Check assert at the end of process

    Command:
        spin -run urns.pml
*/

active proctype urns()
{
    int blacks = 150;
    int whites = 75;
    do
    :: (blacks + whites > 1) ->
        if
        ::	(blacks >= 2) -> blacks--
        ::	(whites >= 2) -> whites = whites - 2; blacks++
        ::	(whites >= 1 && blacks >= 1) -> blacks--;
        fi

    :: else -> break;
    od

    assert(blacks == 0 && whites == 1);
}
\end{lstlisting}

\clearpage

\subsection{Przykład weryfikatora wielowątkowego w języku promela} \label{dodatek:promelaMT}

\begin{lstlisting}[language=Promela,style=myPromela]
/*
    Algorithm: Dijkstra's Mutual exclusion
    CHECK:
       1) Only 1 process can execute crit section code
    Program:
       1) Never Claim is used. This will check assert in every step of program
    Command:
        spin -T -u10000 -g mutual_exclusion.pml
*/
#define N 8
int values[N]
chan msg[N] = [0] of {int}
int crit = 0
#define START_JOB() crit = crit + 1;
#define END_JOB() crit = crit - 1

active proctype thread_first()
{
    int val
    int id = _pid
    assert(id == 0)
    do
    ::  msg[id] ? val ; if
                        :: (val == values[id]) -> START_JOB(); values[id] = (values[id] + 1) % (N + 1) ; END_JOB();
                        :: else -> skip
                        fi
    ::  msg[id + 1] ! values[id]
    od
}

active[N - 1] proctype thread()
{
    int val
    int id = _pid
    assert(id > 0 && id < N)
    do
    :: msg[id] ? val ;  if
                        :: (val != values[id]) -> START_JOB(); values[id] = val ; END_JOB();
                        :: else -> skip
                        fi
    :: msg[(id + 1) % N] ! values[id]
    od
}
never
{
   do
   :: assert(crit <= 1)
   od
}
\end{lstlisting}

Powyższy model sprawdza algorytm wykluczania się opracowany przez \href{http://citeseerx.ist.psu.edu/viewdoc/download?doi=10.1.1.93.314&rep=rep1&type=pdf}{Dijkstre}. Algorytm ten zawiera dodatkowo piękną własność samostabilizacji, czyli jeśli kiedyś przypadkowo (np. przez bitflip) dojdzie do błędu w systemie. Czyli do sytuacji, w której więcej niż 1 wątek będzie korzystał z sekcji krytycznej to w skończonej liczbie kroków algorytm sam się naprawi i dalej będzie bronił sekcji krytycznej. \\
Jego schemat działania jest bardzo prosty. Ustawiamy wątki w pierścień według ich $id$. Zamieniamy wtedy ich $id$ na liczby porządkowe, tak aby sąsiad z prawej miał większe $id$ tylko o 1, a sąsiad z lewej mniejsze o 1. \\
Wyróżniamy 2 rodzaje wątków, wątek pierwszy i pozostałe. Wątek pierwszy rusza się wtedy i tylko wtedy gdy wątek ostatni posiada taką samą wartość $val$. Zwiększa on wtedy swoją wartość $val$ o 1. Pozostałe wątki ruszają się tylko gdy ich wartość $val$ jest różna od wartości $val$ poprzednika (lewego sąsiada). Wtedy po zakończeniu ruchu ustawiają $val$ na wartość swojego sąsiada.