\epigraph{Nie mogę nikogo niczego nauczyć. \\ Mogę tylko sprawić, że zaczną myśleć.}{\textit{Sokrates}}
\section{Wprowadzenie}

Ten dokument zawiera listę zadań (podobną do tych, które można spotkać na studiach), które zostały uznane za wartościowe i takie, które polepszą umiejętności programistyczne.

\textbf{Zadania podzielone są na 3 kategorie:}
\begin{itemize}
    \item \textbf{inżynierskie} - polegają na zaimplementowaniu małej części systemu np. alokatora. Są to zadania najbliższe temu co robimy w pracy.
    \item \textbf{eksperymentalne} - polegają na zaimplementowaniu kilku różnych wariantów danego algorytmu lub systemu, a następnie przeprowadzenia eksperymentów pokazujących wady i zalety wykorzystanych metod.
    \item \textbf{algorytmiczne} - polegają na rozwiązaniu zadania tekstowego skrywającego podstawowy problem informatyczny, np. Vertex Cover.
    %\item \textbf{optymalizacyjne} - polegają na rozwiązaniu problemu optymalizującego przy użyciu wybranego solvera LP, IP, MiP np glpsol.
\end{itemize}

%Zadanie powinno zostać rozwiązane samodzielnie, bez ingerencji pozostałych członków zespołu. W momencie podjęcia się zadania, osoba w dokumencie zespołu wpisuje przy zadaniu swoje imię oraz status: $claimed$. \\
%\textbf{UWAGA: Jedna osoba może mieć tylko 1 zadanie w statusie $claimed$}.

\textbf{Rozwiązanie zadania powinno zawierać następujące elementy:}
\begin{itemize}
    \item \textbf{Opis problemu} - należy przedstawić problemy (informatyki) jakie należy rozwiązać aby uzyskać działające rozwiązanie zadania. Np: Alokator - fragmentacja, bin packing problem. Głównie tyczy się to zadań algorytmicznych.
    \item \textbf{Kod źródłowy} - należy udostępnić wszystkim swój kod źródłowy i krótko go omówić na prezentacji zadania. Projekt powinien zawierać czytelną strukturę plików i katalogów, a także automatyczny system budowania: make, cmake.
    Przykładowy Makefile można znaleźć w dodatku \ref{dodatek:makefile}.
    \item \textbf{Działający przykład} - autor implementacji powinien przygotować przykład, który odpali przy wszystkich podczas prezentacji. Wyjątkiem są eksperymenty, które czasami trzeba przeprowadzić offline. Wtedy przykład możemy sobie darować.
    \item \textbf{Schemat działania} - podczas prezentacji należy przedstawić schemat działania wybranego algorytmu, najlepiej wykonać na sucho (bez kodu) swój algorytm na wybranym przykładzie. Ten punkt najlepiej wykonać w formie prezentacji i rysunków.
    \item \textbf{Benchmarki} - jeśli zadanie prosi o eksperymenty należy przygotować i udostępnić kod do przeprowadzania eksperymentów i generowania wyników. Tak aby każdy mógł pobawić się implementacją i sprawdzić wpływ zmian na zachowanie się algorytmu. Wyniki eksperymentów mogą być przedstawione w formie wydruku na konsoli lub wykresów. Spróbuj napisać skrypt do generowania wykresów. Możesz użyć do tego dowolnego języka jak python czy julia lub narzędzia gnuplot.
    Do zmierzenia czasu wykonywania się kodu w języku C możesz użyć makra umieszczonego w dodatku \ref{dodatek:benchmark}.
\end{itemize}
