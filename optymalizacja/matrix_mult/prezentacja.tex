\documentclass[xcolor={dvipsnames}]{beamer}
\setbeamercovered{transparent}
 
 \usepackage{listings}
\usepackage{ucs}
\usepackage[utf8x]{inputenc}
\usepackage[polish]{babel}
\usepackage{polski}
\usepackage[T1]{fontenc}
\usepackage{amsfonts}
\usepackage{animate}
\usepackage{amsmath}

\usetheme{Madrid}
\urlstyle{same}

\setbeamercovered{invisible}

\AtBeginSection[]{
  \begin{frame}
  \vfill
  \centering
  \begin{beamercolorbox}[sep=8pt,center,shadow=true,rounded=true]{title}
    \usebeamerfont{title}\insertsectionhead\par%
  \end{beamercolorbox}
  \vfill
  \end{frame}
}

\lstset{
  	literate=	{ą}{{\k a}}1
  		     	{Ą}{{\k A}}1
           		{ż}{{\. z}}1
          	 	{Ż}{{\. Z}}1
           		{ź}{{\' z}}1
           		{Ź}{{\' Z}}1
           		{ć}{{\' c}}1
           		{Ć}{{\' C}}1
           		{ę}{{\k e}}1
           		{Ę}{{\k E}}1
           		{ó}{{\' o}}1
           		{Ó}{{\' O}}1
           		{ń}{{\' n}}1
           		{Ń}{{\' N}}1
           		{ś}{{\' s}}1
           		{Ś}{{\' S}}1
           		{ł}{{\l}}1
           		{Ł}{{\L}}1
}

% C lang style
\lstloadlanguages{C}

\lstdefinestyle{C99}
{
	language=C,
 	basicstyle=\ttfamily,
	keywordstyle=\color{blue}\ttfamily,
	stringstyle=\color{red}\ttfamily,
	commentstyle=\color{OliveGreen}\ttfamily,
	numbers=left,
	stepnumber=1,
	firstnumber=1,
    numberfirstline=true,
    numbersep=10pt,
	breaklines=true,
	breakautoindent=false,
	breakatwhitespace=false,
	%postbreak=\space,
	tabsize=2,
	frame=single,
	linewidth=12cm,
	xleftmargin=0.5cm,
	showspaces=false,
	showstringspaces=false,
	extendedchars=true,
	backgroundcolor=\color{white},
	%postbreak=\mbox{\textcolor{red}{$\hookrightarrow$}\space},
 	morekeywords={size_t, ssize_t, uint8_t, uint16_t, uint32_t, uint64_t, int8_t, int16_t, int32_t, int64_t, restrict, inline, typeof, sizeof, ptrdiff_t, __attribute__}
 }
 
 % Wiekszy font do autora
 \setbeamerfont{author}{size=\fontsize{18pt}{20pt}}
 
\author{Michal Kukowski}
\title[Optymalizacja - mnożenie macierzy]{Mnożenie macierzy}
\date{} % bez daty

\begin{document}

\begin{frame}
	\titlepage
\end{frame}

\begin{frame}{Mnożenie macierzy}
\LARGE
	\[
		A = \begin{pmatrix}
			a_{11} & a_{12} & a_{13} \\
			a_{21} & a_{22} & a_{23} \\
			a_{31} & a_{32} & a_{33}
		\end{pmatrix}
		\quad
		B = \begin{pmatrix}
			b_{11} & b_{12} \\
			b_{21} & b_{22}  \\
			b_{31} & b_{32} 
		\end{pmatrix}
	\]
	\vspace{0.3cm}
	\[
		C = A \times B = \quad
		\begin{pmatrix}
			c_{11} & c_{12} \\
			c_{21} & c_{22}  \\
			c_{31} & c_{32} 
		\end{pmatrix}
	\]
\end{frame}

\begin{frame}{Wierszowo - kolumnowe mnożenie}
\LARGE
	\begin{block}{}
		To podejście jest najszęściej stosowane. W szkole uczą nas tego mnożenia macierzy, ponieważ w jednym kroku (używając jednego równania), możemy obliczyć każdy element $c_{ij}$.
		Stosujemy wzór:
		\[ c_{ij}=\displaystyle\sum_{k=1}^n a_{ik} * b_{kj} \]
	\end{block}
\end{frame}

\begin{frame}{Liczymy}
	\LARGE
	\[
		A = \begin{pmatrix}
			a_{11} & a_{12} & a_{13} \\
			a_{21} & a_{22} & a_{23} \\
			a_{31} & a_{32} & a_{33}
		\end{pmatrix}
		\quad
		B = \begin{pmatrix}
			b_{11} & b_{12} \\
			b_{21} & b_{22}  \\
			b_{31} & b_{32} 
		\end{pmatrix}
	\]
	\vspace{0.3cm}
	\[  c_{11} = a_{11} * b_{11} +  a_{12} * b_{21} + a_{13} * b_{31}  \]
	\[  c_{12} = a_{11} * b_{12} +  a_{12} * b_{22} + a_{13} * b_{32}  \]
\end{frame}

\begin{frame}{Wierszowo - wierszowe mnożenie}
\LARGE
	\[  c_{11} = a_{11} * b_{11} +  a_{12} * b_{21} + a_{13} * b_{31}  \]
	\[  c_{12} = a_{11} * b_{12} +  a_{12} * b_{22} + a_{13} * b_{32}  \]
	Rozpiszmy to inaczej:
	\[  c_{11} = 0 + a_{11} * b_{11} \qquad c_{12} = 0 + a_{11} * b_{12} \]
	\[  c_{11} = c_{11} +  a_{12} * b_{21} \qquad  c_{12} = c_{12} + a_{12} * b_{22} \]
	\[  c_{11} = c_{11} +  a_{13} * b_{31} \qquad  c_{12} = c_{12} + a_{13} * b_{32} \]
\end{frame}

\begin{frame}{Wierszowo - wierszowe mnożenie}
	\Large
	\begin{block}{}
		\begin{itemize}
			\item W jedym ruchu liczymy tylko część wyniku (pola $c_{ij}$)
			\item Pętla wewnętrzna leci po $j$, czyli $i$ stałe, $j$ zmienne
			\item Podczas liczenia elemetów $c$, element macierzy $A$ jest stały (wewnętrzna pętla)
			\item Przechodzenie każdej macierzy jest po drugim wymiarze, czyli tak jak układa się pamięć (cache-friendly)
			\item Odwołujemy się wiele razy to elementu macierzy $C$
		\end{itemize}
	\end{block}
\end{frame}

\end{document}
